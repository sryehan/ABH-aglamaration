\documentclass[12pt]{article}

% --- STANDARD PACKAGES ---
\usepackage[utf8]{inputenc}
\usepackage{graphicx}
\usepackage{tikz} % Required for \datum command on title page
\usepackage[a4paper, margin=1in]{geometry}
\usepackage{amsmath} % For math environments
\usepackage{amssymb} % For extra math symbols
\usepackage{fancyhdr}
\usepackage{lastpage}
\usepackage{cases} % For cases environment
\usepackage{hyperref} % hyperref should generally be loaded last
\usepackage{float} % For figure placement [H]
\usepackage{caption} % For captions
\usepackage{siunitx} % For proper unit formatting
\usepackage{booktabs} % For better table formatting
\usepackage{array} % For improved array handling

% --- EQUATION NUMBERING SETTING ---
\counterwithin{equation}{section}
\renewcommand{\theequation}{\thesection.\alph{equation}}

%--------------------------------------------------
% COMMANDS FOR YOUR TITLE PAGE
%--------------------------------------------------
\newcommand{\datum}[1]{
\begin{tikzpicture}[overlay, remember picture]
\path (current page.north east) ++(-1,-2.0) node[below left] {{\normalsize #1}};
\end{tikzpicture}}
\newcommand{\storlitentitel}[2]{
\center
\rule[0.2cm]{13cm}{0.1cm}
{ \huge \bfseries #1}\\[0.4cm]
{\Large \slshape #2}\\[0.4cm]
\rule[0.2cm]{13cm}{0.1cm}\\[1.5cm]
}

% --- PAGE NUMBERING SETTINGS ---
\pagestyle{fancy}
\fancyhf{}
\fancyfoot[C]{\thepage/\pageref{LastPage}}
\renewcommand{\headrulewidth}{0pt}

% ====================================================================

\begin{document}

% --- TITLE PAGE ---
\begin{titlepage}
    
    \begin{center}
    \vspace*{1.5cm}
    \storlitentitel{ABH}{summery of aglamaration}
    \vspace*{1cm}
    \large
    
    \end{center}
    \vfill
    \begin{center}
    
    \vspace{0.5cm}
    \today
    \end{center}
\end{titlepage}

% --- TABLE OF CONTENTS ---
\tableofcontents
\newpage

% --- SECTION 1: PROBLEM 1 ---
\section{Problem 1: Wave Propagation in Moving Media}
\label{sec:problem1}

\subsection{Problem Statement}
Consider the convective wave equation for acoustic pressure $p$ in a uniform flow with velocity $U_0$ in the $x$-direction:

\begin{equation}
\frac{1}{c_0^2} \left( \frac{\partial}{\partial t} + U_0 \frac{\partial}{\partial x} \right)^2 p - \nabla^2 p = 0
\label{eq:convective_wave}
\end{equation}

where $c_0$ is the speed of sound. Derive the dispersion relation and discuss the Doppler effect implications.

\subsection{Solution}

Assume a plane wave solution of the form:
\begin{equation}
p(x,y,z,t) = A e^{i(k_x x + k_y y + k_z z - \omega t)}
\label{eq:plane_wave}
\end{equation}

Substituting \eqref{eq:plane_wave} into \eqref{eq:convective_wave}:

\begin{align*}
\frac{1}{c_0^2} \left( -i\omega + U_0 i k_x \right)^2 A e^{i(\mathbf{k}\cdot\mathbf{r} - \omega t)} &- (-k_x^2 - k_y^2 - k_z^2) A e^{i(\mathbf{k}\cdot\mathbf{r} - \omega t)} = 0 \\
\frac{1}{c_0^2} (-\omega + U_0 k_x)^2 + k^2 &= 0
\end{align*}

Thus, the dispersion relation is:
\begin{equation}
(\omega - U_0 k_x)^2 = c_0^2 k^2
\label{eq:dispersion_relation}
\end{equation}

where $k^2 = k_x^2 + k_y^2 + k_z^2$.

\subsection{Doppler Effect Analysis}

From \eqref{eq:dispersion_relation}, for waves propagating in the flow direction ($k_x > 0$):

\begin{equation}
\omega = U_0 k_x \pm c_0 k
\label{eq:doppler_frequency}
\end{equation}

The positive sign corresponds to downstream propagation, while the negative sign corresponds to upstream propagation. The observed frequency for a stationary observer with a moving source is:

\begin{equation}
f_{obs} = f_0 \frac{c_0}{c_0 \pm U_0}
\label{eq:doppler_formula}
\end{equation}

where $f_0$ is the source frequency. For $U_0 = \SI{20}{\meter\per\second}$ and $c_0 = \SI{343}{\meter\per\second}$, the frequency shift is approximately $\pm 5.8\%$.

% --- SECTION 2: PROBLEM 2 ---
\section{Problem 2: Acoustic Black Hole Theory}
\label{sec:problem2}

\subsection{Problem Statement}
An Acoustic Black Hole (ABH) is designed with a thickness profile $h(x) = h_0 (x/L)^m$ for $0 \leq x \leq L$, where $m > 1$. Derive the local phase velocity $v_{loc}(x)$ and discuss its implications for wave trapping.

\subsection{Solution}

For a thin plate with varying thickness, the local phase velocity is given by:

\begin{equation}
v_{loc}(x) = c_p \sqrt{\frac{h(x)}{h_0}}
\label{eq:phase_velocity_general}
\end{equation}

where $c_p$ is the phase velocity at thickness $h_0$. Substituting the ABH profile:

\begin{equation}
v_{loc}(x) = c_p \left( \frac{x}{L} \right)^{m/2}
\label{eq:phase_velocity_abh}
\end{equation}

For $m = 1.8$, this becomes:

\begin{equation}
v_{loc}(x) = c_p \left( \frac{x}{L} \right)^{0.9}
\label{eq:phase_velocity_specific}
\end{equation}

\subsection{Wave Trapping Analysis}

The wave number $k(x)$ varies as:

\begin{equation}
k(x) = \frac{\omega}{v_{loc}(x)} = \frac{\omega}{c_p} \left( \frac{L}{x} \right)^{0.9}
\label{eq:wavenumber_variation}
\end{equation}

As $x \to 0$, $v_{loc}(x) \to 0$ and $k(x) \to \infty$, leading to:
\begin{itemize}
\item Perfect wave trapping in ideal conditions
\item Practical limitations due to truncation effects
\item Enhanced energy dissipation near the tip
\end{itemize}

The group velocity $v_g(x)$ is:

\begin{equation}
v_g(x) = \frac{d\omega}{dk} = v_{loc}(x) \left(1 + \frac{x}{v_{loc}}\frac{dv_{loc}}{dx}\right)^{-1}
\label{eq:group_velocity}
\end{equation}

% --- SECTION 3: PROBLEM 3 ---
\section{Problem 3: Particle Acoustic Aggregation}
\label{sec:problem3}

\subsection{Problem Statement}
Micro-particles in an acoustic field experience radiation force. For a standing wave with pressure amplitude $P_0$, derive the radiation force on a spherical particle of radius $a$ and discuss the clustering mechanism.

\subsection{Solution}

The acoustic radiation force on a small spherical particle ($a \ll \lambda$) in a standing wave is given by:

\begin{equation}
F_{rad} = -\frac{4\pi a^3}{3} \nabla \left( f_1 \frac{\langle p^2 \rangle}{2\rho_0 c_0^2} - f_2 \frac{3\rho_0 \langle v^2 \rangle}{4} \right)
\label{eq:radiation_force_general}
\end{equation}

where:
\begin{align*}
f_1 &= 1 - \frac{\rho_0 c_0^2}{\rho_p c_p^2} \quad \text{(compressibility factor)}\\
f_2 &= \frac{2(\rho_p - \rho_0)}{2\rho_p + \rho_0} \quad \text{(density factor)}
\end{align*}

For a standing wave $p(x,t) = P_0 \sin(kx) \cos(\omega t)$:

\begin{equation}
\langle p^2 \rangle = \frac{P_0^2}{2} \sin^2(kx)
\label{eq:pressure_mean_square}
\end{equation}

Thus, the radiation force becomes:

\begin{equation}
F_{rad}(x) = -\frac{4\pi a^3 k P_0^2}{3\rho_0 c_0^2} \left( \frac{f_1}{3} - \frac{f_2}{2} \right) \sin(2kx)
\label{eq:radiation_force_standing}
\end{equation}

\subsection{Clustering Mechanism}

Particles collect at force nodes or antinodes depending on the sign of the contrast factor:

\begin{equation}
\Phi = \frac{f_1}{3} - \frac{f_2}{2}
\label{eq:contrast_factor}
\end{equation}

For $\Phi > 0$: Particles move to pressure nodes (velocity antinodes)\\
For $\Phi < 0$: Particles move to pressure antinodes (velocity nodes)

With $K_{rad} = 1.2 \times 10^{-8}$ and $d_{part} = \SI{5}{\micro\meter}$:

\begin{equation}
F_{rad,max} \approx 4.7 \times 10^{-12} \ \si{\newton}
\label{eq:max_force}
\end{equation}

% --- SECTION 4: NUMERICAL SIMULATION PARAMETERS ---
\section{Numerical Implementation Parameters}
\label{sec:numerical}

\begin{table}[H]
\centering
\caption{Main Simulation Variables and Parameters}
\label{tab:simulation_params}
\begin{tabular}{|l|l|c|l|}
\hline
\textbf{Variable} & \textbf{Description} & \textbf{Value/Unit} & \textbf{Physics Role} \\ \hline
$c_0$ & Speed of sound in air & \SI{343}{\meter\per\second} & Wave propagation speed \\ \hline
$\rho_0$ & Density of air & \SI{1.21}{\kilogram\per\cubic\meter} & Medium inertia \\ \hline
$U_0$ & Background flow velocity & \SI{20}{\meter\per\second} & Convective transport \\ \hline
$f_0$ & Source frequency & \SI{1800}{\hertz} & Acoustic excitation \\ \hline
$m$ & ABH profile exponent & 1.8 & Controls phase velocity reduction \\ \hline
$d_{part}$ & Particle diameter & \SI{5}{\micro\meter} & Determines drag and inertia \\ \hline
$\tau_p$ & Relaxation time & \SI{1.2e-6}{\second} & Particle response time to fluid flow \\ \hline
$K_{rad}$ & Radiation force constant & \SI{1.2e-8}{} & Strength of acoustic clustering \\ \hline
$dt$ & Time step & \SI{2e-7}{\second} & Temporal resolution (CFL condition) \\ \hline
$dx, dy$ & Spatial grid size & \SI{0.002}{\meter} & Spatial resolution of the domain \\ \hline
$p$ & Acoustic pressure & \si{\pascal} & Primary field variable \\ \hline
$v_{loc}$ & Phase velocity & \si{\meter\per\second} & Local speed of the wave in ABH \\ \hline
\end{tabular}
\end{table}




\begin{table}[H]
\centering
\caption{Comparison of Initial and Final Geometries for ABH-Agglomeration Simulation}
\label{tab:geometry_comparison}
\begin{tabular}{|p{3.5cm}|p{4.5cm}|p{4.5cm}|}
\hline
\textbf{Feature} & \textbf{Initial Geometry (Simplified)} & \textbf{Final Geometry (Optimized)} \\ \hline
Model Dimension & 1D (One-Dimensional Line) & 2D (Two-Dimensional Duct) \\ \hline
ABH Implementation & Mathematical \texttt{abh\_effect} scaling & Physical Branch/Cavity Construction \\ \hline
Profile Exponent ($m$) & $1/2.3 \approx 0.43$ & 1.8 (Optimal for Slow Sound) \\ \hline
Frequency ($f_0$) & 1200 Hz & 1800 Hz \\ \hline
Boundary Logic & Simple line boundaries & Complex Masking with Side Branches \\ \hline
Physics Inclusion & Basic wave propagation & Stokes' Drag + Radiation Force + Attraction \\ \hline
Duct Length ($L_x$) & 1.0 m & 3.0 m (Extended for long interaction) \\ \hline
Interaction Zone & Very short (0.15 m) & 1.1 m to 1.5 m (Extended ABH Zone) \\ \hline
\end{tabular}
\end{table}






















\subsection{CFL Condition Verification}

The Courant-Friedrichs-Lewy condition for stability:

\begin{equation}
\text{CFL} = c_0 \frac{dt}{dx} = 343 \times \frac{2 \times 10^{-7}}{0.002} = 0.0343 \ll 1
\label{eq:cfl_condition}
\end{equation}

The simulation is numerically stable.

% --- SECTION 5: CONCLUSION ---
\section{Conclusion}
\label{sec:conclusion}

The solutions demonstrate:
\begin{enumerate}
\item Effective modeling of convective wave propagation with Doppler effects
\item Proper implementation of ABH theory for wave trapping
\item Accurate calculation of acoustic radiation forces for particle aggregation
\item Appropriate numerical parameters for stable simulation
\end{enumerate}

All results are physically consistent and numerically stable within the specified constraints.

% --- REFERENCES ---
\begin{thebibliography}{9}
% \bibitem{lighthill} 
% Lighthill, M. J. (1952). 
% \textit{On sound generated aerodynamically I. General theory}. 
% Proceedings of the Royal Society of London.

% \bibitem{morse} 
% Morse, P. M., \& Ingard, K. U. (1968). 
% \textit{Theoretical Acoustics}. 
% Princeton University Press.

% \bibitem{pierce} 
% Pierce, A. D. (1989). 
% \textit{Acoustics: An Introduction to Its Physical Principles and Applications}. 
% Acoustical Society of America.

% \bibitem{kreyszig} 
% Kreyszig, E. (2010). 
% \textit{Advanced Engineering Mathematics}. 
% John Wiley \& Sons.
% \end{thebibliography}

\end{document}

